% Generated by Sphinx.
\def\sphinxdocclass{report}
\documentclass[letterpaper,10pt,english]{sphinxmanual}
\usepackage[utf8]{inputenc}
\DeclareUnicodeCharacter{00A0}{\nobreakspace}
\usepackage[T1]{fontenc}
\usepackage{babel}
\usepackage{times}
\usepackage[Bjarne]{fncychap}
\usepackage{longtable}
\usepackage{sphinx}
\usepackage{multirow}


\title{pyBL Documentation}
\date{September 13, 2013}
\release{0.1.0}
\author{Arman Arkilic}
\newcommand{\sphinxlogo}{}
\renewcommand{\releasename}{Release}
\makeindex

\newcommand\PYGZbs{\char`\\}
\newcommand\PYGZob{\char`\{}
\newcommand\PYGZcb{\char`\}}

\begin{document}

\maketitle
\tableofcontents
\phantomsection\label{index::doc}


Contents:


\chapter{Introduction}
\label{Introduction:introduction}\label{Introduction:welcome-to-pybl-s-documentation}\label{Introduction::doc}
\begin{notice}{warning}{Warning:}
This version of \textbf{NSLS2 Beamline Python Scripting Environment} is the ongoing development version.In case you have concerns or suggestions, please contact
Arman Arkilic

Brookhaven National Laboratory

National Synchrotron Light Source-II

LS Controls

\href{mailto:arkilic@bnl.gov}{arkilic@bnl.gov}
\end{notice}


\chapter{Project Description:}
\label{Introduction:project-description}\begin{quote}

National Synchrotron Light Source 2 will be replacing BNL's current light source NSLS. NSLS2 is a state of the art, medium-energy electron storage rin(3 billion EV). This command line application and its API provides a Python scripting environment for NSLS2 beamlines.Users can generate their own macros, configurations for hardware control and data processing in the beamlines.This version of the software is primarily for XRay diffraction beamlines with 6 circles (4S+2D).
\end{quote}


\chapter{Installation}
\label{Installation:installation}\label{Installation::doc}
\begin{notice}{warning}{Warning:}
This version of the tutorial requires user to have sudo rights on the machine it is being installed. In case you are running a local version of Python under your \$HOME directory, please contact: \href{mailto:arkilic@bnl.gov}{arkilic@bnl.gov}
\end{notice}


\section{Step 1}
\label{Installation:step-1}
\textbf{Obtain dependencies required for this software:}

\textbf{EPICS:}
Please refer to EPICS Installation Instructions below written by Michael Davidsaver:

NSLS-II Controls Package Repository

This repository contains installable software packages in the Debian Linux format (.deb). The software packaged includes the EPICS distributed control framework, the RTEMS real-time operating system, and related packages. Our goal is to provide a fully functional EPICS environment for users and developers.

Packages in this repository were developed (and are used) by the NSLS-II Controls Group. However, no NSLS-II specific modification have been made. They are also being used by several experimental beamlines at the NSLS.

Currently all packages are built only for Debian Stable (wheezy) on i386 and amd64. However, they can be used with most Debian derived distributions (ie. Ubuntu). At present the limiting factors are time, and lack of a good system to automatically build a series of dependent packages.

Status

Complete listing of repository contents by release point.

Debian Release  NSLS-II Release

Wheezy (7.x)    N/A     N/A     2013B   wheezy  Latest

Squeeze (6.x)   N/A     2012A   2013B   squeeze

Lenny (5.x)     epicsdeb10      2012A   N/A     lenny

\textbf{Getting Started}

Setting up a Debian Stable (wheezy) system to use the repository:
Download the repository signing key: \href{http://epics.nsls2.bnl.gov/debian/repo-key.pub}{http://epics.nsls2.bnl.gov/debian/repo-key.pub}
Add the key to the APT keyring with the following command.

\# apt-key add repo-key.pub

Add the following lines to: /etc/apt/sources.list Also ensure that you include the `contrib' on `non-free' sections of the main repository (for Open Motif needed by EDM)
deb \href{http://epics.nsls2.bnl.gov/debian/}{http://epics.nsls2.bnl.gov/debian/} wheezy main contrib
deb-src \href{http://epics.nsls2.bnl.gov/debian/}{http://epics.nsls2.bnl.gov/debian/} wheezy main contrib

(Remember the space between debian/ and wheezy)
Fetch the list of packages.

\# apt-get update

Install packages.

\# apt-get install {[}see list below...{]}
Note: BNL on-site users remember to configure APT to use the web proxy.

Useful groups of packages to install

Basic EPICS Development Environment

epics-dev build-essential

Linux IOC Dev. Environment (additional)

epics-synapps-dev epics-iocstats-dev visualdct openjdk-6-jdk sysv-rc-softioc

Operator console

edm edm-synapps edm-iocstats striptool alh

As an example

\# apt-get install edm edm-synapps
...

\$ export EDMDATAFILES=.:/usr/lib/epics/op/edl

\$ edm -m `P=somename:,M=axis1' motorx\_all

This has installed the EDM display manager and all the panels which are part of the synApps distribution. For a full list of installable packages see the package listing page.

Parital List of Packaged Software

Some package which will probably be of interest.

Name    Description

Clients

edm     The Extensible Display Manager

alh     The EPICS Alarm Handler

striptool       Strip-chart plotting tool

python-cothread Simple and powerful Python language bindings for EPICS Channel Access clients

epics-catools   Command Line tools (caget, caput, camonitor, ...)

Linux softIOCs

epics-dev       EPICS Base headers and utilities.

epics-synapps-dev       synApps headers and utilities

procserv        Process Server. For running softIoc as daemons.

sysv-rc-softioc SysV style script to start softIocs automatically at boot.

RTEMS IOCs

rtems-gesys-mvme3100    RTEMS Generic System Application (for Emerson MVME3100)

epics-synapps-mvme3100  synApps headers and utilities (for RTEMS on MVME3100)

RTEMS

Currently only the following architectures and Board Support Packages are included. If you use EPICS with a BSP which is included in the RTEMS distribution, but not listed here then let us know.

Architecture    BSP name

i386    pc386

powerpc mvme2100

powerpc mvme2307

powerpc mvme3100

powerpc mvme5500

Releases

Releases are made by making a copy of the current (stable) repository and giving it a name. Once a release is made it will receive only bug-fixes.

Name    Date    Debian Release  EPICS Base      synApps

wheezy/2013A    August 2013     Wheezy (7.0)    3.14.12.3       20130320

squeeze/2013A   August 2013     Squeeze (6.0)   3.14.12.3       20130320

squeeze/2012A   Febuary 2012    Squeeze (6.0)   3.14.11 20111025

lenny/2012A     Febuary 2012    Lenny (5.0)     3.14.11 20111025

epicsdeb10      May 2010        Lenny (5.0)     3.14.10 5.4.1

Building Packages from Source

All source code used to produce the precompiled binary packages in this repository is also present as source packages. These can be obtained and built in two ways.

All binary packages in this repository are built in a clean snapshot of a virtual machine (QEmu or VirualBox). This is done to verify build dependecies and to guard against unintended dependecies.

Release Source

Simply request the source package from the repository.

\$ mkdir dpack

\$ cd dpack

\$ sudo apt-get build-dep epics-base

\$ apt-get source epics-base

\$ cd cd epics-base-3.14.10

\$ debuild -us -uc

\$ cd ..

\$ sudo dpkg -i \emph{.deb}

Versioned Source

The source and patches of all packages are tracked with the Git version control system. These Git repositories use Pristine-tar to store the tarball. Allowing them to be entirely self contained. The git-buildpackage tool can be used to control the build process.

\$ mkdir dpack

\$ cd dpack

\$ mkdir tmp

\$ git clone \href{http://pubweb.bnl.gov/~mdavidsaver/git/dpack/epics-base.git}{http://pubweb.bnl.gov/\textasciitilde{}mdavidsaver/git/dpack/epics-base.git}

\$ cd epics-base

\$ git-buildpackage --git-pristine-tar --git-export-dir=\$PWD/../tmp

All Git repositories are available here or here for anonymous read-only access.

BNL Proxy

For computers on-site, APT needs to be configured to use the http proxy. This can be set in the file /etc/apt/apt.conf:

Acquire::http::Proxy::mirror.bnl.gov ``DIRECT'';

Acquire::http::Proxy  ``http://192.168.1.130:3128'';

Sets the default proxy, but bypasses it for the local Debian mirror. Remember that this will be overridden by the http\_proxy environment variable.

\textbf{PYTHON2.7 and PYTHON-DEV(required for python-cothread):}

\$sudo apt-get install python2.7

\$sudo apt-get install python-dev

\textbf{PYTHON-COTHREAD:}

\$sudo apt-get install python-cothread

\textbf{PYTHON-REQUESTS:}

\$sudo apt-get install python-requests

\textbf{NUMPY and SCIPY:}

\$sudo apt-get install python2.7-numpy

\$sudo apt-get install python2.7-scipy

\textbf{IPYTHON:}

\begin{notice}{warning}{Warning:}
Make sure that ipython is version .11+
\end{notice}

\$sudo apt-get install ipython


\section{Step 2}
\label{Installation:step-2}
\textbf{Clone git repository:}
\$git clone \href{https://github.com/arkilic/pyBL.git}{https://github.com/arkilic/pyBL.git}


\section{Step 3}
\label{Installation:step-3}
\begin{notice}{warning}{Warning:}
Perform this step only if you do not have active motor EPICS motor PVs!
\end{notice}

\textbf{Build simulated EPICS motor record under motorSim folder inside git repository}

\$cd motorSim

\$make clean

\$make

Once the installation is complete, confirm simulated motors are running properly:

./startSimMotorEdm.sh

./startSimMotor.sh

\textbf{Alternatively, if you have EPICS motor PVs modify the PVs inside pyBL.conf:}

{[}diffractometer\_config{]}

name=X11

geometry=SixCircle

engine=you

tag=default\_tag

author=pyBL

pv1=test:m1

pv2=test:m2

pv3=test:m3

pv4=test:m4

pv5=test:m5

pv6=test:m6


\section{Step 4}
\label{Installation:step-4}
\begin{notice}{warning}{Warning:}
Please validate the PV names for the simulated motors match the PV names inside pyBL.conf file.
\end{notice}

By modifying the pyBL.conf file, you can select the geometry, calculation engine, angle names etc...
Under the \$TOP(cloned git repo)

\$./runPyBL


\chapter{Manual}
\label{Manual:module-commands}\label{Manual:manual}\label{Manual::doc}\index{commands (module)}

\section{Commands}
\label{Manual:commands}
Python Beamline Scripting environment for NSLS2 beamlines provides users with routines handling hardware control, experimental logging, reciprocal space calculation and several other services that deals with image processing. The following commands are provided as of version 0.1.0 and are subject to change. Please use an up-to-date version of this code and documentation if you would like to benefit from full-capability.
\index{addref() (in module commands)}

\begin{fulllineitems}
\phantomsection\label{Manual:commands.addref}\pysiglinewithargsret{\code{commands.}\bfcode{addref}}{\emph{*args}}{}
addref -- add reflection interactively
addref {[}h k l{]} \{`tag'\} -- add reflection with current position and energy
addref {[}h k l{]} (p1,p2...pN) energy \{`tag'\} -- add arbitrary reflection

\end{fulllineitems}

\index{allhkl() (in module commands)}

\begin{fulllineitems}
\phantomsection\label{Manual:commands.allhkl}\pysiglinewithargsret{\code{commands.}\bfcode{allhkl}}{\emph{hkl}, \emph{wavelength=None}}{}
allhkl {[}h k l{]} -- print all hkl solutions ignoring limits

\end{fulllineitems}

\index{angles\_to\_hkl() (in module commands)}

\begin{fulllineitems}
\phantomsection\label{Manual:commands.angles_to_hkl}\pysiglinewithargsret{\code{commands.}\bfcode{angles\_to\_hkl}}{\emph{angleTuple}, \emph{energy=None}}{}
Converts a set of diffractometer angles to an hkl position
Usage:
((h, k, l), paramDict)=angles\_to\_hkl(self, (a1, a2,aN), energy=None)

\end{fulllineitems}

\index{c2th() (in module commands)}

\begin{fulllineitems}
\phantomsection\label{Manual:commands.c2th}\pysiglinewithargsret{\code{commands.}\bfcode{c2th}}{\emph{hkl}, \emph{en=None}}{}
c2th {[}h k l{]}  -- calculate two-theta angle for reflection

\end{fulllineitems}

\index{checkub() (in module commands)}

\begin{fulllineitems}
\phantomsection\label{Manual:commands.checkub}\pysiglinewithargsret{\code{commands.}\bfcode{checkub}}{}{}
checkub -- show calculated and entered hkl values for reflections.

\end{fulllineitems}

\index{con() (in module commands)}

\begin{fulllineitems}
\phantomsection\label{Manual:commands.con}\pysiglinewithargsret{\code{commands.}\bfcode{con}}{\emph{*args}}{}
con -- list available constraints and values
con \textless{}name\textgreater{} \{val\} -- constrains and optionally sets one constraint
con \textless{}name\textgreater{} \{val\} \textless{}name\textgreater{} \{val\} \textless{}name\textgreater{} \{val\} -- clears and then fully constrains

Select three constraints using `con' and `uncon'. Choose up to one
from each of the sample and detector columns and up to three from
the sample column.

Not all constraint combinations are currently available:
\begin{quote}

1 x samp:              all 80 of 80
2 x samp and 1 x ref:  chi \& phi
\begin{quote}

mu \& eta
chi=90 \& mu=0 (2.5 of 6)
\end{quote}

2 x samp and 1 x det:  0 of 6
3 x samp:              eta, chi \& phi (1 of 4)
\end{quote}

See also `uncon'

\end{fulllineitems}

\index{delref() (in module commands)}

\begin{fulllineitems}
\phantomsection\label{Manual:commands.delref}\pysiglinewithargsret{\code{commands.}\bfcode{delref}}{\emph{num}}{}
delref num -- deletes a reflection (numbered from 1)

\end{fulllineitems}

\index{editref() (in module commands)}

\begin{fulllineitems}
\phantomsection\label{Manual:commands.editref}\pysiglinewithargsret{\code{commands.}\bfcode{editref}}{\emph{num}}{}
editref num -- interactively edit a reflection.

\end{fulllineitems}

\index{getLogLevel() (in module commands)}

\begin{fulllineitems}
\phantomsection\label{Manual:commands.getLogLevel}\pysiglinewithargsret{\code{commands.}\bfcode{getLogLevel}}{}{}
Returns the threshold of Python Logging Instances. User can set the level of detail
for ``diffractometer'' log files:
\begin{quote}

Level       When it is used
DEBUG       Detailed information, typically of interest only when diagnosing problems.
INFO        Confirmation that things are working as expected.
WARNING     An indication that something unexpected happened, or indicative of some problem in the near future (e.g. disk space low). The software is still working as expected.
ERROR       Due to a more serious problem, the software has not been able to perform some function.
CRITICAL    A serious error, indicating that the program itself may be unable to continue running.
\end{quote}

\end{fulllineitems}

\index{get\_low\_limit() (in module commands)}

\begin{fulllineitems}
\phantomsection\label{Manual:commands.get_low_limit}\pysiglinewithargsret{\code{commands.}\bfcode{get\_low\_limit}}{\emph{name}}{}~\begin{description}
\item[{Usage: }] \leavevmode
get\_low\_limit(name=Angle)

\end{description}

Returns the low limit for a specific angle.

\end{fulllineitems}

\index{hkl\_to\_angles() (in module commands)}

\begin{fulllineitems}
\phantomsection\label{Manual:commands.hkl_to_angles}\pysiglinewithargsret{\code{commands.}\bfcode{hkl\_to\_angles}}{\emph{h}, \emph{k}, \emph{l}, \emph{energy}}{}
Convert a given hkl vector to a set of diffractometer angles

\end{fulllineitems}

\index{listub() (in module commands)}

\begin{fulllineitems}
\phantomsection\label{Manual:commands.listub}\pysiglinewithargsret{\code{commands.}\bfcode{listub}}{}{}
listub -- list the ub calculations available to load.

\end{fulllineitems}

\index{loadub() (in module commands)}

\begin{fulllineitems}
\phantomsection\label{Manual:commands.loadub}\pysiglinewithargsret{\code{commands.}\bfcode{loadub}}{\emph{name\_or\_num}}{}
loadub \{`name'{\color{red}\bfseries{}\textbar{}}num\} -- load an existing ub calculation

\end{fulllineitems}

\index{move() (in module commands)}

\begin{fulllineitems}
\phantomsection\label{Manual:commands.move}\pysiglinewithargsret{\code{commands.}\bfcode{move}}{\emph{angle}, \emph{position}}{}
Move a single motor to a user designated location. For multiple motors, one can 
write their own custom macros by utilizing position() or move().
Position is suggested if user defined motion involves several motors moving 
in a coherent fashion.Unlike mainstream beamline hardware motion control
and XRay diffraction experiment software, this tool allows user to move 
multiple motors simulatenously. Users can also use this python scripting environment
in order to control other hardware alongside motors also simulatenously. 
Usage:
\begin{quote}

move(angle,position)
\end{quote}

\end{fulllineitems}

\index{newub() (in module commands)}

\begin{fulllineitems}
\phantomsection\label{Manual:commands.newub}\pysiglinewithargsret{\code{commands.}\bfcode{newub}}{\emph{name=None}}{}
newub \{`name'\} -- start a new ub calculation name

\end{fulllineitems}

\index{position() (in module commands)}

\begin{fulllineitems}
\phantomsection\label{Manual:commands.position}\pysiglinewithargsret{\code{commands.}\bfcode{position}}{\emph{**args}}{}
Users can define their own angle-value pair dictionaries.The motors will be moved to these positions 
if the defined positions are within motors' hardware limits
Usage:
\begin{quote}

position(\{`angle\#1':value1,'angle2':value2,...\})
\end{quote}

\end{fulllineitems}

\index{saveubas() (in module commands)}

\begin{fulllineitems}
\phantomsection\label{Manual:commands.saveubas}\pysiglinewithargsret{\code{commands.}\bfcode{saveubas}}{\emph{name}}{}
saveubas `name' -- save the ub calculation with a new name

\end{fulllineitems}

\index{setLogLevel() (in module commands)}

\begin{fulllineitems}
\phantomsection\label{Manual:commands.setLogLevel}\pysiglinewithargsret{\code{commands.}\bfcode{setLogLevel}}{\emph{level}}{}
Sets the threshold of Python Logging Instances. User can set the level of detail
for diffractometer log files:
\begin{quote}

Level       When it is used
DEBUG       Detailed information, typically of interest only when diagnosing problems.
INFO        Confirmation that things are working as expected.
WARNING     An indication that something unexpected happened, or indicative of some problem in the near future (e.g. disk space low). The software is still working as expected.
ERROR       Due to a more serious problem, the software has not been able to perform some function.
CRITICAL    A serious error, indicating that the program itself may be unable to continue running.
\end{quote}

\end{fulllineitems}

\index{setName() (in module commands)}

\begin{fulllineitems}
\phantomsection\label{Manual:commands.setName}\pysiglinewithargsret{\code{commands.}\bfcode{setName}}{\emph{name}}{}
setName--sets the diffractometer configuration name.

\end{fulllineitems}

\index{setlat() (in module commands)}

\begin{fulllineitems}
\phantomsection\label{Manual:commands.setlat}\pysiglinewithargsret{\code{commands.}\bfcode{setlat}}{\emph{name=None}, \emph{*args}}{}
setlat  -- interactively enter lattice parameters (Angstroms and Deg)
setlat name a -- assumes cubic
setlat name a b -- assumes tetragonal
setlat name a b c -- assumes ortho
setlat name a b c gamma -- assumes mon/hex with gam not equal to 90
setlat name a b c alpha beta gamma -- arbitrary

\end{fulllineitems}

\index{setu() (in module commands)}

\begin{fulllineitems}
\phantomsection\label{Manual:commands.setu}\pysiglinewithargsret{\code{commands.}\bfcode{setu}}{\emph{U=None}}{}
setu \{((,,),(,,),(,,))\} -- manually set u matrix

\end{fulllineitems}

\index{setub() (in module commands)}

\begin{fulllineitems}
\phantomsection\label{Manual:commands.setub}\pysiglinewithargsret{\code{commands.}\bfcode{setub}}{\emph{UB=None}}{}
setub \{((,,),(,,),(,,))\} -- manually set ub matrix

\end{fulllineitems}

\index{showref() (in module commands)}

\begin{fulllineitems}
\phantomsection\label{Manual:commands.showref}\pysiglinewithargsret{\code{commands.}\bfcode{showref}}{}{}
showref -- shows full reflection list

\end{fulllineitems}

\index{swapref() (in module commands)}

\begin{fulllineitems}
\phantomsection\label{Manual:commands.swapref}\pysiglinewithargsret{\code{commands.}\bfcode{swapref}}{\emph{num1=None}, \emph{num2=None}}{}
swapref -- swaps first two reflections used for calulating U matrix
swapref num1 num2 -- swaps two reflections (numbered from 1)

\end{fulllineitems}

\index{trialub() (in module commands)}

\begin{fulllineitems}
\phantomsection\label{Manual:commands.trialub}\pysiglinewithargsret{\code{commands.}\bfcode{trialub}}{}{}
trialub -- (re)calculate u matrix from ref1 only (check carefully).

\end{fulllineitems}

\index{ub() (in module commands)}

\begin{fulllineitems}
\phantomsection\label{Manual:commands.ub}\pysiglinewithargsret{\code{commands.}\bfcode{ub}}{}{}
ub -- show the complete state of the ub calculation

\end{fulllineitems}

\index{wavelength() (in module commands)}

\begin{fulllineitems}
\phantomsection\label{Manual:commands.wavelength}\pysiglinewithargsret{\code{commands.}\bfcode{wavelength}}{\emph{value=None}}{}
wavelength(value=None)-- sets the wavelength for the reciprocal space calculations

\end{fulllineitems}

\index{where() (in module commands)}

\begin{fulllineitems}
\phantomsection\label{Manual:commands.where}\pysiglinewithargsret{\code{commands.}\bfcode{where}}{}{}
where()--Returns the all the motor positions and current hkl coordinates

\end{fulllineitems}



\chapter{Developer Manual}
\label{Manual:developer-manual}\label{Manual:module-Diffractometer}\index{Diffractometer (module)}
Diffractometer and hardware used for XRAY Diffraction experiments are treated as objects with multiple attributes by this code.As a result, diffractometer objects are customizable for each beamline/user. DiffCalc (by Rob Walton-Diamond Light Source) is the heart of the reciprocal space calculation engine and in order to perform reciprocal space calculations, this software creates custom diffractometer and hardware-dependent instances and maps the attributes of these instances (axis names, motor positions, limits, etc...) to DiffCalc objects. In other words, this code uses DiffCalc API without wrapping DiffCalc code, leaving DiffCalc standalone for future updates.

\textbf{As of v0.1:}

As Diffcalc documentation also states, DiffCalc core calculation code works with a six-circle geometry.It supports four-circle modes, where two circles are fixed @ zero, five-circle modes, where one circle is fixed and the last is used to keep surface normal in the horizontal lab plane,and six-circle modes where the surface normal is kept parallel to the omega (theta) axis.For each of these there are five variants: the angle of the incoming or outgoing beam to the crystal surface can be fixed the incoming and outgoing angles can be made equal, phi can be fixed,or the azimuthal angle about the momentum-transfer vector can be fixed.The azimuthal variants still need some testing and likely development.                                       
DiffCalc does not directly move motors. It is only a reciprocal space calculator. Hardware motion is provided through this software(via EPICS services). Angles stand for the axes(circles) of the diffractometer.EPICS Process Variables(PVs) are assigned to angle instances.These PVs are provided by EPICS IOC and EPICS asyn driver.For more details on this, please check EPICS motor record documentation(\href{http://www.aps.anl.gov/bcda/synApps/motor/}{http://www.aps.anl.gov/bcda/synApps/motor/}). Flexible nature of EPICS applications allows users to add custom hardware on their own,making this software a multi-hardware-platform application.
\index{Angle (class in Diffractometer)}

\begin{fulllineitems}
\phantomsection\label{Manual:Diffractometer.Angle}\pysiglinewithargsret{\strong{class }\code{Diffractometer.}\bfcode{Angle}}{\emph{name}, \emph{value}, \emph{geometry}, \emph{positiveLimit}, \emph{negativeLimit}, \emph{author}}{}
Hey!

\end{fulllineitems}

\index{Diffractometer (class in Diffractometer)}

\begin{fulllineitems}
\phantomsection\label{Manual:Diffractometer.Diffractometer}\pysiglinewithargsret{\strong{class }\code{Diffractometer.}\bfcode{Diffractometer}}{\emph{name}, \emph{geometry}, \emph{engine}, \emph{tag}, \emph{author}}{}
Constructor-Name, tag, author, angle list(axes names) are chosen by the user based on their preferences or standards. Diffractometer expects to get either FourCircle or SixCircle options as geometry. There are 3 engines supported by this software: `you', `vlieg', `willmott'. The latest and fastest of the three is `you', however, users can choose one engine over another based on their application. Hardware attribute is a placeholder for DiffCalc Hardware Adapter. As of this version, this software utilizes DummyHardwareAdapter. However, in the future versions, this will be replaced with a custom HardwareAdapter instance as we will determine preferences and standards in NSLS2 XRay Diffraction Beamline
\index{basicSetup() (Diffractometer.Diffractometer method)}

\begin{fulllineitems}
\phantomsection\label{Manual:Diffractometer.Diffractometer.basicSetup}\pysiglinewithargsret{\bfcode{basicSetup}}{\emph{hardwareAdapter}, \emph{**params}}{}
Sets up a basic diffractometer with default values. These values can be changed by using native functions such as
someAngle.setName(),someAngle.setpositive () can be used. If this is not the preference as this requires setting up too many parameters,
diffcalc.config.advancedSetup() provides a cleaner/more organized way to set up a custom diffractometer by utilizing dictionaries.
self.engine=engine   
self.tag='Basic diffractometer configuration'
self.author='default'
self.defaultAngleParam=\{`value':0,
\begin{quote}

`geometry':SixCircle(),
`positiveLimit':180,
`negativeLimit':-180\}
\end{quote}

parameterList=\{`angles','geometry'\}

\end{fulllineitems}

\index{createAngles() (Diffractometer.Diffractometer method)}

\begin{fulllineitems}
\phantomsection\label{Manual:Diffractometer.Diffractometer.createAngles}\pysiglinewithargsret{\bfcode{createAngles}}{\emph{angle}, \emph{Geometry}}{}
Creates Angle instances for a hardware.Each angle instance is created and manipulated
separately. The user has complete control of each circle of a diffractometer.

\end{fulllineitems}

\index{getAngleNames() (Diffractometer.Diffractometer method)}

\begin{fulllineitems}
\phantomsection\label{Manual:Diffractometer.Diffractometer.getAngleNames}\pysiglinewithargsret{\bfcode{getAngleNames}}{}{}
Returns a list of Angle instances that includes all the angles associated with a given diffractometer
angleList is updated after every operation that changes motor positions.

\end{fulllineitems}

\index{getAngleValues() (Diffractometer.Diffractometer method)}

\begin{fulllineitems}
\phantomsection\label{Manual:Diffractometer.Diffractometer.getAngleValues}\pysiglinewithargsret{\bfcode{getAngleValues}}{}{}
Returns a list of Angle values. These values are read from the EPICS motor record
and always refer to actual motor position readings.

\end{fulllineitems}

\index{getClient() (Diffractometer.Diffractometer method)}

\begin{fulllineitems}
\phantomsection\label{Manual:Diffractometer.Diffractometer.getClient}\pysiglinewithargsret{\bfcode{getClient}}{}{}
Returns an Olog Client object that can be used to access several attributes that may be used for searching entries, creating new logbook, tag, property and/or present user with information regarding logs.

\end{fulllineitems}

\index{getDCInstance() (Diffractometer.Diffractometer method)}

\begin{fulllineitems}
\phantomsection\label{Manual:Diffractometer.Diffractometer.getDCInstance}\pysiglinewithargsret{\bfcode{getDCInstance}}{}{}
Returns the DiffCalc instance that a a specific Diffractometer is mapped onto. By using this DiffCalc object, developers can write custom applications that deal directly with DiffCalc objects. This is useful once a custom diffcalc functionality is written inside diffcalc, as it is done under commands.py, developer can create a function under this API that is directly linked to the custom diffcalc function.

\end{fulllineitems}

\index{getEngine() (Diffractometer.Diffractometer method)}

\begin{fulllineitems}
\phantomsection\label{Manual:Diffractometer.Diffractometer.getEngine}\pysiglinewithargsret{\bfcode{getEngine}}{}{}
Returns DiffCalc calculation engine used in order to notify the user. This makes it possible to write applications that use different calculation engines based on different papers(you,vlieg,willmott) and compare recirporcal space/motor positions.

\end{fulllineitems}

\index{getGeometry() (Diffractometer.Diffractometer method)}

\begin{fulllineitems}
\phantomsection\label{Manual:Diffractometer.Diffractometer.getGeometry}\pysiglinewithargsret{\bfcode{getGeometry}}{}{}
Returns diffractometer geometry in string format. The reason behind this is to simplify geometry selection for the user through configuration file. For a custom reciprocal space calculation or geometry, a developer should create custom geometries inside DiffCalc(see DiffCalc Developer Manual) and call these geometries via Diffractometer.setGeometry(). Developer also needs to assure that proper number of motors(Angle instances) are created via Config.py.

\end{fulllineitems}

\index{getHardware() (Diffractometer.Diffractometer method)}

\begin{fulllineitems}
\phantomsection\label{Manual:Diffractometer.Diffractometer.getHardware}\pysiglinewithargsret{\bfcode{getHardware}}{}{}
Returns the hardware used for reciprocal space calculations. This is strictly for diffcalc,
however,Angle names and Angle values are completely in coherence with userAPI.

\end{fulllineitems}

\index{getName() (Diffractometer.Diffractometer method)}

\begin{fulllineitems}
\phantomsection\label{Manual:Diffractometer.Diffractometer.getName}\pysiglinewithargsret{\bfcode{getName}}{}{}
Returns the diffractometer configuration name. This can be used to identify a specific configuration of a diffractometer as this attribute is accessed directly through the configuration file

\end{fulllineitems}

\index{setAnglesforHardware() (Diffractometer.Diffractometer method)}

\begin{fulllineitems}
\phantomsection\label{Manual:Diffractometer.Diffractometer.setAnglesforHardware}\pysiglinewithargsret{\bfcode{setAnglesforHardware}}{\emph{angleList}, \emph{Geometry}}{}
Creates Angle instances for a user defined diffractometer. These angles are going to be used fo setting up 
reciprocal space calculations as well as hardware motion control. 
Each angle instance is assigned to a motor, which provides a coherent structure making it simple to generate 
custom geometries for beamline scientists. This also makes it possible to construct a hardware independent
configuration that is easy to maintain.

\end{fulllineitems}

\index{setClient() (Diffractometer.Diffractometer method)}

\begin{fulllineitems}
\phantomsection\label{Manual:Diffractometer.Diffractometer.setClient}\pysiglinewithargsret{\bfcode{setClient}}{\emph{client}}{}
Sets up an olog client for the given diffractometer configuration. Developers/users can modify this olog client.However, one must be really careful not to lose existing log entries as log entries created have a client field and permissions to these entries that strictly depend on this client.

\end{fulllineitems}

\index{setDCInstance() (Diffractometer.Diffractometer method)}

\begin{fulllineitems}
\phantomsection\label{Manual:Diffractometer.Diffractometer.setDCInstance}\pysiglinewithargsret{\bfcode{setDCInstance}}{}{}
To be modified as the hardware adapter is initiated

\end{fulllineitems}

\index{setEngine() (Diffractometer.Diffractometer method)}

\begin{fulllineitems}
\phantomsection\label{Manual:Diffractometer.Diffractometer.setEngine}\pysiglinewithargsret{\bfcode{setEngine}}{\emph{engine}}{}
Sets the engine used in diffraction experiment. This engine is used in reciprocal space
calculations through diffcalc. 
Supported engines: YOU, WILLMOTT,VLIEG

\end{fulllineitems}

\index{setGeometry() (Diffractometer.Diffractometer method)}

\begin{fulllineitems}
\phantomsection\label{Manual:Diffractometer.Diffractometer.setGeometry}\pysiglinewithargsret{\bfcode{setGeometry}}{\emph{geometry}}{}
Sets a diffractometer's geometry. This geometry is used for both motor control and
and reciprocal space calculations.

\end{fulllineitems}

\index{setHardwareAdapter() (Diffractometer.Diffractometer method)}

\begin{fulllineitems}
\phantomsection\label{Manual:Diffractometer.Diffractometer.setHardwareAdapter}\pysiglinewithargsret{\bfcode{setHardwareAdapter}}{\emph{hardwareAdapter}}{}
Sets up a hardware adapter for DiffCalc calculations.
Available adapters:
\begin{quote}

DummyHardwareAdapter(diffractometerAngleNames)
\begin{description}
\item[{HardwareAdapter(diffractometerAngleNames, }] \leavevmode
defaultCuts=\{\}, 
energyScannableMultiplierToGetKeV=1)

\end{description}
\end{quote}

\end{fulllineitems}


\end{fulllineitems}



\chapter{Developer Manual}
\label{Developer Manual:developer-manual}\label{Developer Manual::doc}\label{Developer Manual:module-Diffractometer}\index{Diffractometer (module)}
Diffractometer and hardware used for XRAY Diffraction experiments are treated as objects with multiple attributes by this code.As a result, diffractometer objects are customizable for each beamline/user. DiffCalc (by Rob Walton-Diamond Light Source) is the heart of the reciprocal space calculation engine and in order to perform reciprocal space calculations, this software creates custom diffractometer and hardware-dependent instances and maps the attributes of these instances (axis names, motor positions, limits, etc...) to DiffCalc objects. In other words, this code uses DiffCalc API without wrapping DiffCalc code, leaving DiffCalc standalone for future updates.

\textbf{As of v0.1:}

As Diffcalc documentation also states, DiffCalc core calculation code works with a six-circle geometry.It supports four-circle modes, where two circles are fixed @ zero, five-circle modes, where one circle is fixed and the last is used to keep surface normal in the horizontal lab plane,and six-circle modes where the surface normal is kept parallel to the omega (theta) axis.For each of these there are five variants: the angle of the incoming or outgoing beam to the crystal surface can be fixed the incoming and outgoing angles can be made equal, phi can be fixed,or the azimuthal angle about the momentum-transfer vector can be fixed.The azimuthal variants still need some testing and likely development.                                       
DiffCalc does not directly move motors. It is only a reciprocal space calculator. Hardware motion is provided through this software(via EPICS services). Angles stand for the axes(circles) of the diffractometer.EPICS Process Variables(PVs) are assigned to angle instances.These PVs are provided by EPICS IOC and EPICS asyn driver.For more details on this, please check EPICS motor record documentation(\href{http://www.aps.anl.gov/bcda/synApps/motor/}{http://www.aps.anl.gov/bcda/synApps/motor/}). Flexible nature of EPICS applications allows users to add custom hardware on their own,making this software a multi-hardware-platform application.
\index{Angle (class in Diffractometer)}

\begin{fulllineitems}
\phantomsection\label{Developer Manual:Diffractometer.Angle}\pysiglinewithargsret{\strong{class }\code{Diffractometer.}\bfcode{Angle}}{\emph{name}, \emph{value}, \emph{geometry}, \emph{positiveLimit}, \emph{negativeLimit}, \emph{author}}{}
Hey!

\end{fulllineitems}

\index{Diffractometer (class in Diffractometer)}

\begin{fulllineitems}
\phantomsection\label{Developer Manual:Diffractometer.Diffractometer}\pysiglinewithargsret{\strong{class }\code{Diffractometer.}\bfcode{Diffractometer}}{\emph{name}, \emph{geometry}, \emph{engine}, \emph{tag}, \emph{author}}{}
Constructor-Name, tag, author, angle list(axes names) are chosen by the user based on their preferences or standards. Diffractometer expects to get either FourCircle or SixCircle options as geometry. There are 3 engines supported by this software: `you', `vlieg', `willmott'. The latest and fastest of the three is `you', however, users can choose one engine over another based on their application. Hardware attribute is a placeholder for DiffCalc Hardware Adapter. As of this version, this software utilizes DummyHardwareAdapter. However, in the future versions, this will be replaced with a custom HardwareAdapter instance as we will determine preferences and standards in NSLS2 XRay Diffraction Beamline
\index{basicSetup() (Diffractometer.Diffractometer method)}

\begin{fulllineitems}
\phantomsection\label{Developer Manual:Diffractometer.Diffractometer.basicSetup}\pysiglinewithargsret{\bfcode{basicSetup}}{\emph{hardwareAdapter}, \emph{**params}}{}
Sets up a basic diffractometer with default values. These values can be changed by using native functions such as
someAngle.setName(),someAngle.setpositive () can be used. If this is not the preference as this requires setting up too many parameters,
diffcalc.config.advancedSetup() provides a cleaner/more organized way to set up a custom diffractometer by utilizing dictionaries.
self.engine=engine   
self.tag='Basic diffractometer configuration'
self.author='default'
self.defaultAngleParam=\{`value':0,
\begin{quote}

`geometry':SixCircle(),
`positiveLimit':180,
`negativeLimit':-180\}
\end{quote}

parameterList=\{`angles','geometry'\}

\end{fulllineitems}

\index{createAngles() (Diffractometer.Diffractometer method)}

\begin{fulllineitems}
\phantomsection\label{Developer Manual:Diffractometer.Diffractometer.createAngles}\pysiglinewithargsret{\bfcode{createAngles}}{\emph{angle}, \emph{Geometry}}{}
Creates Angle instances for a hardware.Each angle instance is created and manipulated
separately. The user has complete control of each circle of a diffractometer.

\end{fulllineitems}

\index{getAngleNames() (Diffractometer.Diffractometer method)}

\begin{fulllineitems}
\phantomsection\label{Developer Manual:Diffractometer.Diffractometer.getAngleNames}\pysiglinewithargsret{\bfcode{getAngleNames}}{}{}
Returns a list of Angle instances that includes all the angles associated with a given diffractometer
angleList is updated after every operation that changes motor positions.

\end{fulllineitems}

\index{getAngleValues() (Diffractometer.Diffractometer method)}

\begin{fulllineitems}
\phantomsection\label{Developer Manual:Diffractometer.Diffractometer.getAngleValues}\pysiglinewithargsret{\bfcode{getAngleValues}}{}{}
Returns a list of Angle values. These values are read from the EPICS motor record
and always refer to actual motor position readings.

\end{fulllineitems}

\index{getClient() (Diffractometer.Diffractometer method)}

\begin{fulllineitems}
\phantomsection\label{Developer Manual:Diffractometer.Diffractometer.getClient}\pysiglinewithargsret{\bfcode{getClient}}{}{}
Returns an Olog Client object that can be used to access several attributes that may be used for searching entries, creating new logbook, tag, property and/or present user with information regarding logs.

\end{fulllineitems}

\index{getDCInstance() (Diffractometer.Diffractometer method)}

\begin{fulllineitems}
\phantomsection\label{Developer Manual:Diffractometer.Diffractometer.getDCInstance}\pysiglinewithargsret{\bfcode{getDCInstance}}{}{}
Returns the DiffCalc instance that a a specific Diffractometer is mapped onto. By using this DiffCalc object, developers can write custom applications that deal directly with DiffCalc objects. This is useful once a custom diffcalc functionality is written inside diffcalc, as it is done under commands.py, developer can create a function under this API that is directly linked to the custom diffcalc function.

\end{fulllineitems}

\index{getEngine() (Diffractometer.Diffractometer method)}

\begin{fulllineitems}
\phantomsection\label{Developer Manual:Diffractometer.Diffractometer.getEngine}\pysiglinewithargsret{\bfcode{getEngine}}{}{}
Returns DiffCalc calculation engine used in order to notify the user. This makes it possible to write applications that use different calculation engines based on different papers(you,vlieg,willmott) and compare recirporcal space/motor positions.

\end{fulllineitems}

\index{getGeometry() (Diffractometer.Diffractometer method)}

\begin{fulllineitems}
\phantomsection\label{Developer Manual:Diffractometer.Diffractometer.getGeometry}\pysiglinewithargsret{\bfcode{getGeometry}}{}{}
Returns diffractometer geometry in string format. The reason behind this is to simplify geometry selection for the user through configuration file. For a custom reciprocal space calculation or geometry, a developer should create custom geometries inside DiffCalc(see DiffCalc Developer Manual) and call these geometries via Diffractometer.setGeometry(). Developer also needs to assure that proper number of motors(Angle instances) are created via Config.py.

\end{fulllineitems}

\index{getHardware() (Diffractometer.Diffractometer method)}

\begin{fulllineitems}
\phantomsection\label{Developer Manual:Diffractometer.Diffractometer.getHardware}\pysiglinewithargsret{\bfcode{getHardware}}{}{}
Returns the hardware used for reciprocal space calculations. This is strictly for diffcalc,
however,Angle names and Angle values are completely in coherence with userAPI.

\end{fulllineitems}

\index{getName() (Diffractometer.Diffractometer method)}

\begin{fulllineitems}
\phantomsection\label{Developer Manual:Diffractometer.Diffractometer.getName}\pysiglinewithargsret{\bfcode{getName}}{}{}
Returns the diffractometer configuration name. This can be used to identify a specific configuration of a diffractometer as this attribute is accessed directly through the configuration file

\end{fulllineitems}

\index{setAnglesforHardware() (Diffractometer.Diffractometer method)}

\begin{fulllineitems}
\phantomsection\label{Developer Manual:Diffractometer.Diffractometer.setAnglesforHardware}\pysiglinewithargsret{\bfcode{setAnglesforHardware}}{\emph{angleList}, \emph{Geometry}}{}
Creates Angle instances for a user defined diffractometer. These angles are going to be used fo setting up 
reciprocal space calculations as well as hardware motion control. 
Each angle instance is assigned to a motor, which provides a coherent structure making it simple to generate 
custom geometries for beamline scientists. This also makes it possible to construct a hardware independent
configuration that is easy to maintain.

\end{fulllineitems}

\index{setClient() (Diffractometer.Diffractometer method)}

\begin{fulllineitems}
\phantomsection\label{Developer Manual:Diffractometer.Diffractometer.setClient}\pysiglinewithargsret{\bfcode{setClient}}{\emph{client}}{}
Sets up an olog client for the given diffractometer configuration. Developers/users can modify this olog client.However, one must be really careful not to lose existing log entries as log entries created have a client field and permissions to these entries that strictly depend on this client.

\end{fulllineitems}

\index{setDCInstance() (Diffractometer.Diffractometer method)}

\begin{fulllineitems}
\phantomsection\label{Developer Manual:Diffractometer.Diffractometer.setDCInstance}\pysiglinewithargsret{\bfcode{setDCInstance}}{}{}
To be modified as the hardware adapter is initiated

\end{fulllineitems}

\index{setEngine() (Diffractometer.Diffractometer method)}

\begin{fulllineitems}
\phantomsection\label{Developer Manual:Diffractometer.Diffractometer.setEngine}\pysiglinewithargsret{\bfcode{setEngine}}{\emph{engine}}{}
Sets the engine used in diffraction experiment. This engine is used in reciprocal space
calculations through diffcalc. 
Supported engines: YOU, WILLMOTT,VLIEG

\end{fulllineitems}

\index{setGeometry() (Diffractometer.Diffractometer method)}

\begin{fulllineitems}
\phantomsection\label{Developer Manual:Diffractometer.Diffractometer.setGeometry}\pysiglinewithargsret{\bfcode{setGeometry}}{\emph{geometry}}{}
Sets a diffractometer's geometry. This geometry is used for both motor control and
and reciprocal space calculations.

\end{fulllineitems}

\index{setHardwareAdapter() (Diffractometer.Diffractometer method)}

\begin{fulllineitems}
\phantomsection\label{Developer Manual:Diffractometer.Diffractometer.setHardwareAdapter}\pysiglinewithargsret{\bfcode{setHardwareAdapter}}{\emph{hardwareAdapter}}{}
Sets up a hardware adapter for DiffCalc calculations.
Available adapters:
\begin{quote}

DummyHardwareAdapter(diffractometerAngleNames)
\begin{description}
\item[{HardwareAdapter(diffractometerAngleNames, }] \leavevmode
defaultCuts=\{\}, 
energyScannableMultiplierToGetKeV=1)

\end{description}
\end{quote}

\end{fulllineitems}


\end{fulllineitems}



\chapter{Indices and tables}
\label{index:indices-and-tables}\begin{itemize}
\item {} 
\emph{genindex}

\item {} 
\emph{modindex}

\item {} 
\emph{search}

\end{itemize}


\renewcommand{\indexname}{Python Module Index}
\begin{theindex}
\def\bigletter#1{{\Large\sffamily#1}\nopagebreak\vspace{1mm}}
\bigletter{c}
\item {\texttt{commands}}, \pageref{Manual:module-commands}
\indexspace
\bigletter{d}
\item {\texttt{Diffractometer}}, \pageref{Manual:module-Diffractometer}
\end{theindex}

\renewcommand{\indexname}{Index}
\printindex
\end{document}
